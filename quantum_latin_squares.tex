\documentclass{iansnotes}

\title{Quantum Latin Squares}
\author{ian.mcloughlin@atu.ie}
\date{Last updated: \today}

\begin{document}
 
\maketitle

\section{Quantum Latin Square}
An $n \times n$ array of elements in $\mathbb{C}^n$ such that each row and each column is an orthonormal basis.
\marginnote{\bibentry{musto2016quantum}}

\section{Example}

\begin{tikzpicture}
  \matrix [matrix of math nodes, nodes={draw,minimum width=29mm}] {
    \ket{0}                               & \ket{1} & \ket{2} & \ket{3} \\
    \frac{1}{\sqrt{2}}(\ket{1} - \ket{2}) & \frac{1}{\sqrt{5}}(i\ket{0} + 2\ket{3}) & \frac{1}{\sqrt{5}}(2\ket{0} + i\ket{3}) & \frac{1}{\sqrt{2}}(\ket{1} + \ket{2}) \\
    \frac{1}{\sqrt{2}}(\ket{1} + \ket{2}) & \frac{1}{\sqrt{5}}(2\ket{0} + i\ket{3}) & \frac{1}{\sqrt{5}}(i\ket{0} + 2\ket{3}) & \frac{1}{\sqrt{2}}(\ket{1} - \ket{2}) \\
    \ket{3}                               & \ket{2} & \ket{1} & \ket{0} \\
  };
\end{tikzpicture}

\begin{marginfigure}[10mm]
  \centering 
\begin{tikzpicture}
  \matrix [matrix of math nodes, draw=black] {
    \ket{0} & \ket{1} & \ket{2} & \ket{3} \\
    \ket{1} & \ket{2} & \ket{3} & \ket{0} \\
    \ket{2} & \ket{3} & \ket{0} & \ket{1} \\
    \ket{3} & \ket{0} & \ket{1} & \ket{2} \\
  };
\end{tikzpicture}
\caption{From Latin Square.}
\end{marginfigure}


\end{document}