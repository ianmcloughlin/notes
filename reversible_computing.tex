\documentclass{iansnotes}

\title{Reversible Computing}
\author{ian.mcloughlin@atu.ie}
\date{Last updated: \today}

\begin{document}
 
\maketitle

\section{Logical Reversibility}

\begin{circuitikz}
  \ctikzset{font=\scriptsize};
  \draw   (0,0)   node[and port]  {and};
\end{circuitikz}
\begin{table}
\begin{tabular}{cc|c}
  $x_1$ & $x_2$ & $\land$ \\
  \midrule
  0 & 0 & 0 \\
  0 & 1 & 0 \\
  1 & 0 & 0 \\
  1 & 1 & 1 \\
\end{tabular}
\end{table}\\[6mm]
\begin{circuitikz}
  \ctikzset{font=\scriptsize};
  \draw   (0,0)   node[not port]  {not};
\end{circuitikz}
\begin{table}
\begin{tabular}{c|c}
  $x$ & $\bar{x}$ \\
  \midrule
  0 & 1 \\
  1 & 0 \\
\end{tabular}
\end{table}

\section{Landauer's Principle}
\marginnote[-10mm]{\bibentry{landauer61}}
\marginnote[10mm]{\bibentry{bennett82}}
\marginnote[10mm]{\bibentry{bennett2003notes}}
$\Delta E \geq kT \ln 2$ \\[4mm]
$k = 1.380649 \times 10^{-23} \textsf{JK}^{-1}$ \\[4mm]
$\Delta E \geq (1.380649 \times 10^{-23}) (293) (0.693147) \approx 2.8 \times 10^{-21}\textsf{J} \textrm{ at } 20^{\circ} \textsf{C}$\footnote{Somewhere in th}

\section{$\textsf{XOR}$}
\begin{circuitikz}
  \ctikzset{font=\scriptsize};
  \draw   (0,0)   node[xor port]  {xor};
\end{circuitikz}
\begin{table}
\begin{tabular}{cc|ccccc}
  $x_1$ & $x_2$ & $\land$ \\
  \midrule
  0 & 0 & 0 \\
  0 & 1 & 1 \\
  1 & 0 & 1 \\
  1 & 1 & 0 \\
\end{tabular}
\end{table}\\[6mm]

\section{$\textsf{CNOT}$}
\begin{circuitikz}
  \ctikzset{font=\scriptsize};
  \draw   (0,0)   node[xor port] (X)  {xor};
  \draw (X.in 1) to ++(-0.3,0) to[short, *-] ++(0,0.8) to[short] ++(2.5,0) node[right] {$x$};
  \draw (X.in 1) to[short] ++(-0.8,0)node[left] {$x_1$};
  \draw (X.in 2) to[short] ++(-0.8,0) node[left] {$x_2$};
  \draw (X.out) to[short] ++(0.6,0) node[right] {$x_1 \oplus x_2$};
\end{circuitikz}\\[6mm]
\begin{table}
\begin{tabular}{cc|cc}
  $x_1$ & $x_2$ & $x_1$ & $x_1 \oplus x_2$ \\
  \midrule
  0 & 0 & 0 & 0 \\
  0 & 1 & 0 & 1 \\
  1 & 0 & 1 & 1 \\
  1 & 1 & 1 & 0 \\
\end{tabular}
\end{table}
\marginnote[-20mm]{$x_1 \oplus x_2$ means the same as $x_1 \vee x_2$.}

\end{document}