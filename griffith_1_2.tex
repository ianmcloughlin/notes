\documentclass{iansnotes}

\title{Griffith Example 1.2}
\author{ian.mcloughlin@atu.ie}
\date{Last updated: \today}

\begin{document}
 
\maketitle

\section{Question}

Suppose someone drops a rock off a cliff oh height $h$.
As it falls, I snap a million photographs, as random intervals.
On each picture I measure the distance the rock has fallen.

What is the average of all there distances?
That is to say, what is the time average of the distance travelled?
\marginnote[-40mm]{\bibentry{griffiths2018}}
 
\section{Distance Formula}
$x(t) = \frac{1}{2} gt^2$\\[4mm]
$g = 9.8067 \textsf{ms}^{-2}$\\[4mm]
\begin{marginfigure}[-20mm]
\begin{tikzpicture}[scale=0.5]
  \begin{axis}[
    axis lines = left,
    xlabel = \(t\),
    ylabel = {\(d = \frac{1}{2}gt^2\)},
  ]
  \addplot[
    domain=0:100,
    samples=1000,
    color=gray
  ]{0.5 * 9.8067 * x^2};
  \end{axis}
\end{tikzpicture}
\end{marginfigure}

\section{Velocity}
$x(t) = \frac{1}{2}gt^2$\\[4mm]
$\frac{dx}{dt} = gt$\\[4mm]
\begin{marginfigure}
  \begin{tikzpicture}[scale=0.5]
  \begin{axis}[
    axis lines = left,
    xlabel = \(t\),
    ylabel = {\(v = gt\)},
  ]
  \addplot[
    domain=0:100,
    samples=1000,
    color=gray
  ]{9.8067 * x};
  \end{axis}
\end{tikzpicture}
\end{marginfigure}

\section{Flight Time}
Height: $h$; \hspace{2mm}Total Flight Time: $T$\\[4mm]
$h = \frac{1}{2}gT^2$ \\[4mm]
$\frac{2h}{g} = T^2$ \\[4mm]
$T = \sqrt{\frac{2h}{g}}$

\section{Random Intervals}

\end{document}